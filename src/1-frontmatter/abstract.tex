% !TeX root = ../../tesi.tex
% !TeX encoding = UTF-8 Unicode
% !TeX spellcheck = it_IT

\begin{abstract}

  Uno dei trend più interessanti del momento è l'analisi e mining dei dati di traiettoria.
  Questa categoria di dati si compone principalmente delle tracce di movimento generate dalle più svariate categorie di dispositivi.
  Una traiettoria può essere interpretata come il cambiamento della posizione di un utente o oggetto nello spazio rispetto al tempo.
  Nell'ambito dell'analisi di traiettorie, le tecniche di clustering possono essere impiegate con diversi obbiettivi, come ad esempio la ricerca delle strade più frequentate o la profilazione degli utenti.
  Altrettante potenzialità sono racchiuse nella ricerca di itemset frequenti su dati di traiettoria.
  A metà tra questi due approcci si colloca l'analisi dei co-movements pattern.
  I pattern di co-movimento identificano quei gruppi di utenti che hanno viaggiato assieme per un certo periodo significativo di tempo.
  La ricerca di questi gruppi può estrarre diverse informazioni, come ad esempio le abitudini di un utente sulla base dei gruppi di appartenenza e dell'orario del giorno o ancora il mezzo di trasporto utilizzato da una certa categoria di utenti. 
  
  Obiettivo del lavoro di questa tesi è l'analisi di due algoritmi per la ricerca di pattern di co-movimento in ambito big data.
  Il primo è SPARE, framework  descritto in letteratura, che permette di ricercare diversi pattern di movimento grazie a un mix delle tecniche di clustering di traiettorie e quelle di mining di itemset frequenti.
  L'altro algoritmo invece è CUTE, nuovo approccio definito e implementato in questo lavoro di tesi.
  CUTE si pone come framework di clustering sovrapposto basato su un insieme di dimensioni personalizzabili, che sfrutta le tecniche di colossal itemset mining per ricercare gruppi di movimento sulle dimensioni specificate.
  La struttura di CUTE è adattabile alla ricerca di pattern di co-movimento specificando le dimensioni spazio temporali come dimensioni su cui eseguire la ricerca.
  
\end{abstract}
