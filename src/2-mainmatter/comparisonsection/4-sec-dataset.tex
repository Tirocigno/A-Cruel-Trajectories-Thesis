Nel processo di testing e confronto tra i due algoritmi sono stati impiegati due diversi dataset.

Il primo dataset utilizzato è Geolife (\cite{GeolifeG31:online}).
Questo dataset è composto da un insieme di traiettorie raccolte in Asia, in particolare in prossimità della città di Pechino.
In Geolife sono presenti 17158 traiettorie distinte, composte da punti contenenti latitudine, longitudine e altitudine.
Il \(91\%\) delle traiettorie nel dataset sono generate da dispositivi aventi tempo di campionamento compreso tra \(1\) e \(5\) secondi.
Il tempo totale coperto da tutte le traiettorie nel dataset è circa quattro anni.

Il secondo dataset impiegato è composto da un insieme di traiettorie registrate nella città di Oldenburg.
Questo conta 1000000 traiettorie, a differenza di quanto detto prima, i punti non sono espressi in coordinate polari, ma in coordinate cartesiane rispetto a un sistema di riferimento custom.
Anche per quanto riguarda la dimensione temporale, il dataset viene fornito con punti aventi una componente temporale relativa a una sequenza di istanti.
La sequenza in questione conta 245 istanti distinti.

La \cref{tab:metrics-dataset} riassume le caratteristiche principali dei dataset impiegati nei test.
Per quanto riguarda il numero di traiettorie, Geolife viene utilizzato sempre nella sua interezza, Oldenburg invece partizionato diversamente a seconda delle situazioni di impiego.


\begin{table}[H]
    \centering
   \begin{tabular}{||c c c c c||}
 \hline
     Dataset & Spazio & Tempo & Punti & Traiettorie\\ [0.4ex] 
 \hline\hline
    Geolife & Coordinate polari & yyyy-mm-gg hh:mm:ss & 18891115 & 17158 \\ 
 \hline
     Oldenburg & Coordinate cartesiane & Sequenziale & 64895840 & 1000000 \\ 
 \hline
\end{tabular}
    \caption{Riassunto delle caratteristiche principali dei due dataset utilizzati}
    \label{tab:metrics-dataset}
\end{table}
