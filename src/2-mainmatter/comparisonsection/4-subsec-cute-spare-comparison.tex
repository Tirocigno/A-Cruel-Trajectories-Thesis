CUTE e SPARE entrambi eseguono una ricerca di pattern di co-movimento.
In tutti e due sono presenti i concetti di lunghezza minima del percorso (\(k, \alpha\)), dimensione minima di un gruppo (\(m, \gamma\)) e massimo gap temporale tra i punti del percorso (\(g, \tau\)).

Parlando poi delle idee di funzionamento dei due algoritmi, in entrambi si utilizzano i concetti di itemset,pruning e monotonicità. 
In CUTE ciò viene realizzato dividendo lo spazio di ricerca in celle, assegnando a ogni cella l'insieme delle traiettorie che transitano nei suoi confini spazio-temporali e eseguendo poi una variante distribuita dell'algoritmo di Carpenter per ricercare cluster di traiettorie.
SPARE invece utilizza un approccio basato su itemset solo nella seconda fase, quella relativa al tempo e formula una propria definizione di monotonicità, diversa dalla frequenza, basata su due parametri \(l\) e \(g\).
Nella prima fase utilizza un clustering spaziale per definire ad ogni istante quali punti sono vicini e quali no.

SPARE prende come ipotesi che per ogni istante della scala temporale, ogni traiettoria abbia al massimo una posizione: questo diventa limitante quando si vuole considerare traiettorie generate da diverse scale di campionamento, poiché l'algoritmo riduce tutte le traiettorie a una scala comune.
Questo processo di semplificazione comporta una perdita di informazioni.

Anche su CUTE è necessario compiere un'operazione di assimilazione a una stessa scala temporale, tuttavia CUTE pone l'enfasi della sua divisione nella creazione di celle con un certo volume nello spazio-tempo.
Dal punto di vista temporale, ogni cella non esprime un istante, ma una durata.
Una traiettoria passa da una cella se transita nei suoi confini anche solo per un istante.
Questo implica che una traiettoria possa muoversi su un range di celle aventi stesso istante temporale.
In questo modo CUTE minimizza la perdita di informazione legata all'adozione di una scala temporale univoca su diverse traiettorie.

Dal punto di vista della vicinanza nello spazio, SPARE è più flessibile sulle forme dei cluster, potendo utilizzare vari algoritmi differenti per la creazione di questi.
Al contrario CUTE ha confini spaziali più rigidi, determinati da una divisione regolare dello spazio.
SPARE però considera la vicinanza solo in termini relativi, trascurando i movimenti eseguiti dal gruppo tra due istanti temporali.
CUTE invece permette di esprimere vincoli sulla contiguità spaziale (\(\epsilon\)) tra le varie celle percorse da un gruppo.

Parlando poi della dimensione temporale, CUTE presenta una gestione omogenea rispetto alla dimensione spaziale, processando quindi spazio e tempo assieme, tuttavia l'algoritmo mostra i suoi limiti in corrispondenza di bucket temporali nell'ordine dei secondi o pochi minuti, a causa dell'aumento del numero di celle. 
SPARE invece ha buone performance anche con bucket temporali nell'ordine dei secondi, inoltre permette una maggiore espressività nei vincoli temporali grazie al vincolo \(l\).
Dall'altro lato però CUTE è in grado di gestire scale temporale cicliche come i giorni della settimana o le ore del giorno, mentre SPARE è limitato a una scala temporale monotona.

CUTE consente inoltre di processare altre dimensioni senza nessuna ulteriore espansione, mentre SPARE è pensato esclusivamente solo per processare dati spazio-temporali.

Trattando infine i risultati, SPARE genera cluster basati su itemset frequenti massimali, CUTE invece su frequenti chiusi. 
CUTE quindi, a parità di configurazione, genererà sempre un numero maggiore di risultati rispetto a SPARE.