Anche utilizzando definizioni di similarità personalizzate sul problema, i risultati fra i due cluster sono comunque molto diversi fra di loro.

Sicuramente entrambi gli algoritmi formalizzano la ricerca di pattern di co-movimento con alcuni elementi in comune, tuttavia l'approccio alla ricerca e la natura stessa degli itemset individuati porta a escludere una casistica reale in cui gli algoritmi individuano esattamente gli stessi cluster.
È però possibile tentare di massimizzare l'intersezione tra i risultati di questi due algoritmi, sempre tenendo presente che le cardinalità di CUTE sono di molto superiori a quelle di SPARE.

In teoria a parità di risultati sul clustering spaziale dovrebbe garantire quanto meno la totale sovrapposizione totale dei risultati di SPARE su quelli di CUTE.
Questo perché il tempo viene processato nello stesso modo dai due algoritmi, mentre invece lo spazio no.
Tuttavia trovare questo punto di intersezione è praticamente impossibile, SPARE valuta i cluster ad ogni istante temporale, definendo potenzialmente una configurazione spaziale diversa ad ogni istante.
La divisione effettuata da CUTE invece è fissa in tutti gli istanti temporali.

Sia CUTE che SPARE dispongono poi di vincoli che l'altro non non è in grado di coprire.
Questi devono essere necessariamente rilassati durante il confronto, per non far divergere ulteriormente i risultati.
Ciò però implica che il potenziale espressivo di questi parametri sia perso nella ricerca.

In definitiva CUTE e SPARE pongono due approcci che possono convergere parzialmente nella ricerca di pattern di co-movimento.
Allo stesso tempo tuttavia portano un contributo unico alla ricerca che viene minimizzato quando si vuole far convergere i risultati.