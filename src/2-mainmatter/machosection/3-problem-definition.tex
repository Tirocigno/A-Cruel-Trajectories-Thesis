Prima di concentrarsi sui singoli passi di CUTE, occorre definire la formulazione dell'algoritmo.
Innanzitutto si pone necessario riprendere i concetti di itemset, supporto e transazione descritti nella~\cref{sec:fim}
e di traiettoria grezza e astrazione di traiettoria presentati nella~\cref{subsec:trajectory-definition}.

Rispetto a questi concetti, la formulazione di CUTE è descritta dai seguenti parametri:

\begin{enumerate}

  \item \textbf{Sistema di riferimento \(n\)-dimensionale}.
  Dato lo spazio di ricerca definito da \(n\) dimensioni, questo viene diviso in ipercubi sulla base del sistema di riferimento impiegato.
  Ogni ipercubo così individuato è definito quanto.
  Le traiettorie saranno inizialmente espresse come insiemi di quanti, successivamente ogni quanto diverrà una transazione durante la fase di itemset mining.
  
  \item \textbf{Funzione di distanza tra quanti}.
  Dati due quanti \(q_{1}, q_{2}\) espressi su \(k\) dimensioni, \(d_q(q_{1}, q_{2})\) determina la distanza tra i due considerando tutte le possibili dimensioni.
  Il risultato di questa operazione è un vettore \(k\) dimensionale.
  
  \item \textbf{Soglia di distanza}.
   Dati due quanti \(q_{1}, q_{2}\)  espressi su \(k\) dimensioni e una funzione di distanza \(d_q\) si definisce \(\delta\) come un vettore di \(k\) valori ciascuno relativo a una dimensione.
   \(q_{1}, q_{2}\) si dicono vicini se per ciascuna dimensione la loro distanza è minore della soglia presente in \(\delta\).
   
  \item \textbf{Dimensione minima}. Questo parametro \(\gamma \) riguarda il numero minimo di item all'interno di ogni itemset restituito in output alla fine dell'algoritmo.
  
  \item \textbf{Supporto minimo}. La frequenza minima \(\alpha \) oltre cui ogni itemset deve apparire all'interno dell'insieme delle transazioni.

\end{enumerate}

Questa definizione è generica e include qualunque possibile dimensione.

Rispetto al lavoro svolto in questo documento di tesi, è opportuno fornire la formulazione di CUTE per il riconoscimento di pattern di movimento sulla base delle dimensioni spazio-temporali.
I quanti definiti sono quindi cubi a tre dimensioni: latitudine, longitudine e tempo.
Inoltre, per facilitare il confronto successivo con SPARE, vengono formulate separatamente la dimensione spaziale e quella temporale.
Tuttavia questa è solo una procedura di forma, l'algoritmo non altera la sua originale formulazione né le sue logiche di funzionamento.
Di seguito vengono presentati i parametri di CUTE adattato alla ricerca di pattern di movimento:

\begin{enumerate}

\item \textbf{Sistema di riferimento tridimensionale}.
Nella ricerca di pattern di co-movimento sono impiegate tre dimensioni, due spaziali e una temporale: la latitudine la longitudine ed il tempo.

\begin{itemize}
     \item \textbf{Lato spaziale del quanto \(s\)}.
  Definisce la dimensione di latitudine e longitudine coperta da ogni quanto.
  
  \item \textbf{Lato temporale del quanto \(t\)}.
  Definisce la durata temporale di ogni quanto.
\end{itemize}

\item \textbf{Funzione di distanza tra quanti}.
Vengono impiegate due differenti funzioni di distanza:
\begin{itemize}

 \item \textbf{Funzione di distanza spaziale tra quanti \(d_s\)}.
  Dati due quanti \(q_{1}, q_{2}\), \(d_s(q_{1}, q_{2})\) determina la distanza spaziale tra i due.
  
   \item \textbf{Funzione di distanza temporali tra quanti \(d_t\)}.
  Dati due quanti \(q_{1}, q_{2}\), \(d_t(q_{1}, q_{2})\) determina la distanza temporale tra due quanti.
    
\end{itemize}
\item \textbf{Soglia di distanza}.
Anche qui le soglie temporali vengono distinte:
\begin{itemize}
    \item \textbf{Soglia di distanza spaziale \(\epsilon\)}.
   Dati due punti \(p_{1}, p_{2}\) e una funzione di distanza spaziale \(d_{s}\),
  si definisce \(\epsilon \) come la soglia sotto la quale due punti si considerano vicini.
  
  \item \textbf{Soglia di distanza temporale \(\tau \)}. Analogamente a quanto detto sopra, il parametro \(\tau \) permette di specificare
  una soglia massima per la distanza tra due punti nel tempo.
\end{itemize}

\end{enumerate}

Con questa formulazione, risulta possibile per CUTE affrontare il problema
della ricerca di co-movement pattern, descritto nella \cref{sec:comovement-definition}.
Scendendo nel dettaglio, \(\gamma \) esprime la dimensione minima di un gruppo considerato interessante, mentre
\(~\alpha \) il numero minimo di istanti spazio-temporali in cui tutti i membri del gruppo sono considerati vicini.
\(\epsilon \) rappresenta il criterio di vicinanza spaziale tra due punti, \(\tau \) invece permette la ricerca dei diversi pattern di movimento:
con \(\tau \) = 1 si ha un vincolo stringente sulla continuità temporale, andando quindi a ricercare pattern flock,
dall'altra parte con un valore uguale a \(\infty \), si rilassa al massimo la continuità temporale, ottenendo
così dei risultati classificabili come swarm. Infine con qualunque valore intermedio tra 1 e  \(\infty \),
CUTE esegue la ricerca di pattern platoon degeneri, ovvero con vincolo \(l\) posto uguale a 1 e \(\tau\) = \(g\).