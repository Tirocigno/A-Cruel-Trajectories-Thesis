Prima di concentrarsi sui singoli passi di \textit{Cu.Te}, occorre definire la formulazione dell'algoritmo.
Innanzitutto si pone necessario riprendere i concetti di itemset, supporto e transazione descritti nella~\cref{sec:frequent-itemset-mining}
e di traiettoria grezza e astrazione di traiettoria presentati nella~\cref{subsec:problem:trajectorydata}.

A queste definizioni va aggiunto il concetto di sistema di riferimento: si definisce un sistema di riferimento un insieme di
quanti completo e continuo all'interno di una regione spazio-temporale.
I sistemi di riferimento sono fondamentali nel determinare le dimensioni spazio-temporali dei punti delle varie traiettorie,
un esempio su tutti può essere una scala di riferimento espressa in coordinate polari.
Tuttavia è possibile definire anche sistemi di riferimento che utilizzano metriche diverse dalle coordinate sopracitate,
ottenendo così diversi effetti sulla rappresentazione dei dati.

Per quanto riguarda invece l'aspetto collegato alla ricerca di itemset,
in \textit{Cu.Te} viene introdotto il concetto di \textit{Cohesion} o coesione: dato un itemset \textit{I} = \{ \textit{i\textsubscript{1}},\ldots, \textit{i\textsubscript{n}}\}
avente supporto definito come \textit{s}, si definisca la funzione \textit{allTransaction(i)} %chktex 36
 che dato un item \textit{i} restituisce l'insieme delle
transazioni che lo supportano. A questo punto è possibile definire la coesione con la seguente formulazione:

\[ coh(I) = \frac{\lvert \lvert allTransaction(\textit{i\textsubscript{1}})~\cap ,\ldots, \cap~allTransaction(\textit{i\textsubscript{1}}) \rvert\rvert}{\lvert \lvert allTransaction(\textit{i\textsubscript{1}})~\cup ,\ldots, \cup~allTransaction(\textit{i\textsubscript{1}}) \rvert\rvert} \]

Considerando che il numeratore corrisponde esattamente a \textit{s}, allora è possibile esprimere l'equazione come:

\[ coh(I) = \frac{s}{\lvert \lvert allTransaction(\textit{i\textsubscript{1}})~\cup ,\ldots, \cup~allTransaction(\textit{i\textsubscript{1}}) \rvert\rvert} \]

La nuova misura permette di misurare la compattezza di un itemset rispetto agli elementi che lo compongono.
La coesione si pone quindi come meccanica di filtraggio aggiuntiva al supporto:
mentre quest'ultimo è una misura in termini assoluti di frequenza, la coesione risulta una metrica relativa alla frequenza dei singoli elementi.
Ne segue che un alto valore di coesione andrà a scartare quegli itemset i cui item risultano molto sparsi all'interno delle transazioni,
mantenendo invece quelli che risultano più compatti.
