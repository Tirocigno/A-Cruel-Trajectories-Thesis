Prima di concentrarsi sui singoli passi di \textit{Cu.Te}, occorre definire la formulazione dell'algoritmo.
Innanzitutto si pone necessario riprendere i concetti di itemset, supporto e transazione descritti nella~\cref{sec:problem:frequent-itemset-mining}
e di traiettoria grezza e astrazione di traiettoria presentati nella~\cref{subsec:problem:trajectorydata}.

A queste definizioni va aggiunto il concetto di sistema di riferimento: si definisce un sistema di riferimento un insieme di
quanti completo e continuo all'interno di una regione spazio-temporale.
I sistemi di riferimento sono fondamentali nel determinare le dimensioni spazio-temporali dei punti delle varie traiettorie,
un esempio su tutti può essere una scala di riferimento espressa in coordinate polari.
Tuttavia è possibile definire anche sistemi di riferimento che utilizzano metriche diverse dalle coordinate sopracitate,
ottenendo così diversi effetti sulla rappresentazione dei dati.

Per quanto riguarda invece l'aspetto collegato alla ricerca di itemset,
in \textit{Cu.Te} viene introdotto il concetto di \textit{Cohesion} o coesione: dato un itemset \textit{I} = \{ \textit{i\textsubscript{1}},\ldots, \textit{i\textsubscript{n}}\}
avente supporto definito come \textit{s}, si definisca la funzione \textit{allTransaction(i)} %chktex 36
 che dato un item \textit{i} restituisce l'insieme delle
transazioni che lo supportano. A questo punto è possibile definire la coesione con la seguente formulazione:

\[ coh(I) = \frac{\lvert \lvert allTransaction(\textit{i\textsubscript{1}})~\cap ,\ldots, \cap~allTransaction(\textit{i\textsubscript{n}}) \rvert\rvert}{\lvert \lvert allTransaction(\textit{i\textsubscript{1}})~\cup ,\ldots, \cup~allTransaction(\textit{i\textsubscript{n}}) \rvert\rvert} \]

Considerando che il numeratore corrisponde esattamente a \textit{s}, allora è possibile esprimere l'equazione come:

\[ coh(I) = \frac{s}{\lvert \lvert allTransaction(\textit{i\textsubscript{1}})~\cup ,\ldots, \cup~allTransaction(\textit{i\textsubscript{n}}) \rvert\rvert} \]

La nuova misura permette di misurare la compattezza di un itemset rispetto agli elementi che lo compongono.
La coesione si pone quindi come meccanica di filtraggio aggiuntiva al supporto:
mentre quest'ultimo è una misura in termini assoluti di frequenza, la coesione risulta una metrica relativa alla frequenza dei singoli elementi.
Ne segue che un alto valore di coesione andrà a scartare quegli itemset i cui item risultano molto sparsi all'interno delle transazioni,
mantenendo invece quelli che risultano più compatti.

Una volta definiti questi due concetti, è opportuno elencare e definire i parametri
utilizzati da \textit{Cu.Te}.

\begin{itemize}

  \item Quanto spaziale \textit{s}:
  Definisce la dimemsione spaziale del sistema di riferimento, in relazione alle coordinate polari di ogni punto.

  \item Quanto temporale \textit{t}:
  Definisce la dimensione temporale del sistema di riferimento rispetto al tempo misurato in secondi di ogni punto di una traiettoria.
  \item Soglia di distanza spaziale\(~\epsilon \):
  Dati due punti \textit{p\textsubscript{1}, p\textsubscript{2}} e una funzione di distanza spaziale \textit{d\textsubscript{s}},
  si definisce \(~\epsilon \) come la soglia sotto la quale due punti si considerano vicini.
  \item Soglia di distanza temporale\(~\tau \): analogamente a quanto detto sopra, il parametro \(~\tau \) permette di specificare
  una soglia massima per la distanza tra due punti nel tempo.
  \item Dimensione minima\(~\gamma \): Questo parametro riguarda il numero minimo di item all'interno di ogni itemset restituito in output alla fine dell'algoritmo.
  \item Supporto minimo\(~\alpha \): La frequenza minima oltre cui ogni itemset deve apparire all'interno dell'insieme delle transazioni.
  \item Coesione minima\(~\beta \): Come definito sopra, esprime un limite inferiore alla coesione degli itemset prodotti dall'algoritmo.

\end{itemize}

Grazie a questi parametri, risulta possibile per \textit{Cu.Te} affrontare il problema
della ricerca di \textit{comovement-pattern}, descritto nella \cref{sec:problem:comovements-pattern}.
Scendendo nel dettaglio, \(~\gamma \) esprime la dimensione minima di un gruppo considerato interessante, mentre
\(~\alpha \) il numero minimo di istanti spazio-temporali in cui tutti i membri del gruppo sono considerati vicini.
\(~\epsilon \) rappresenta il criterio di vicinanza spaziale tra due punti, \(~\tau \) invece permette la ricerca dei diversi pattern di movimento:
con \(~\tau \) = 1 si ha un vincolo stringente sulla continuità temporale, andando quindi a ricercare pattern \textit{flock},
dall'altra parte con un valore uguale a \(~\infty \), si rilassa al massimo la continuità temporale, ottenendo
così dei risultati classificabili come \textit{Swarm}. Infine con qualunque valore intermedio tra 1 e  \(~\infty \),
\textit{Cu.Te} esegue la ricerca di pattern \textit{Group} degeneri, ovvero con vincolo \textit{L} posto uguale a 1.


