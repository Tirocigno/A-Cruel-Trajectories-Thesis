Una delle principali sfide nel distribuire l'algoritmo di Carpenter è la traduzione dei tre vincoli di pruning, pensati per funzionare in maniera centralizzata.
I primi due sono facilmente riproducibili, ma il terzo è più complicato.
La ragione è che prevede una struttura di memoria centralizzata dove inserire di volta in volta gli itemset valutati.
Ovviamente questo non è realizzabile facilmente in contesto distribuito, in quanto l'accesso in scrittura a risorse centralizzate pone diversi problemi, come ad esempio le corse critiche.
Occorre quindi formulare una nuova definizione per individuare quando un itemset sia già stato valutato o meno.
Intuitivamente, il supporto di un itemset può essere interpretato come un percorso di quanti.
Scopo della ricerca di itemset chiusi è di trovare gli itemset aventi maggior supporto: alla luce di quanto detto sopra, il supporto più grande corrisponde al percorso più lungo.
Avendo le celle un identificatore ordinato, ogni supporto può essere interpretato come un percorso che parte dal quanto con id minore e arriva in quello con supporto maggiore.
Alla luce di ciò, terzo criterio di pruning di Carpenter può essere riformulato come: ``una tupla deve essere mantenuta se questa sta costruendo il percorso più lungo per un certo itemset, altrimenti questa deve essere scartata''.

Analizzando questo nuovo criterio, il vincolo può essere descritto nei seguenti punti: 
\begin{itemize}
    \item Non devono esistere celle con ID minore all'inizio della sequenza che contengono quest'ultima nel loro vicinato.
    Intuitivamente se ciò accadesse, vorrebbe dire che il percorso massimo non comincia dall'attuale prima cella.
    In tal caso infatti esisterebbe un'altra cella che genererà lo stesso percorso che sta venendo considerato ma con supporto totale maggiore.
    Per evitare una ripetizione è necessario quindi scartare l'attuale tupla.
    \item La somma del supporto totale e delle potenziali celle che contengono l'itemset \(I\) deve avere valore maggiore di \(\alpha\).
    Se ciò non fosse vero, \(I\) non potrebbe mai raggiungere la soglia di supporto per essere considerato frequente, di conseguenza 
    sarebbe inutile espandere la tupla nelle iterazioni successive.
    \item La somma del supporto totale, delle potenziali celle che contengono l'itemset \(I\) da esplorare e delle celle antecedenti alla testa della sequenza valide per il supporto devono coprire tutte le celle che supportano \(I\).
    In caso contrario, non sarebbero mai considerate potenziali celle per l'espansione.
    
\end{itemize}
La funzione di filtraggio viene formalizzata nell'~\cref{alg:filter}
\begin{algorithm}[H]
\caption{Filter}\label{alg:filter}
\begin{algorithmic}[1]
\Require $\mathcal{T}$: Insieme delle transazioni, $N$: Vicinato, $\alpha$: soglia di supporto, $\gamma$: soglia di cardinalità.
\Ensure $l$ è frequente e potenzialmente chiuso. 
\If{$l.lcluster.size >= \gamma \wedge l.X.size + l.R.size >= \alpha$} \Comment{La tupla ha una cardinalità valida e un supporto potenzialmente buono}
\State $totalSupport = sup(l.lcluster)$ 
\If{$!N.isEmpty$}                                    \Comment{Se il vicinato è definito, la tupla deve risultare valida rispetto ai vincoli spazio-temporali}
\State $minCellID = l.X.min$
\State $OutlierCells \gets \varnothing$
\ForEach{$ cell \in totalSupport$}
\If{$cell.ID >= min.ID \vee minCellID \notin neighborhood(cell)$}
\State $OutlierCells \cup cell$
\EndIf
\EndFor
\If{$(totalSupport \setminus l.X \setminus l.R \setminus OutlierCells) == \varnothing$} \Comment{Controlla che tutte le celle continue siano dentro X o R}
\State \Return true
\EndIf
\Else  \Comment{Altrimenti controlla che X + R contenga tutte le celle che supportano \(I\)}
\If{$min(totalSupport) \in l.X \wedge (totalSupport \setminus l.X \setminus l.R) == \varnothing $}
\State \Return true
\EndIf
\EndIf
\EndIf
\State \Return false
\end{algorithmic}
\end{algorithm}

I vincoli sopra espressi permettono quindi di esprimere un filtro analogo a quello espresso nell'algoritmo di Carpenter.