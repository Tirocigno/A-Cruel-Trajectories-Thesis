Le tecniche presentate nei capitoli precedenti rappresentano lo stato dell'arte per quanto riguarda il clustering di traiettorie.
In particolare la ricerca dei pattern di movimento presenta un interessante mix tra l'approccio clustering e mining di itemset frequenti.
Il framework SPARE in particolare implementa la ricerca di pattern di co-movimento generici su piattaforma big data.

Tra le possibilità inesplorate della letteratura ci sono le tecniche di mining ad alta dimensionalità.
Queste sono state applicate ai dati di traiettoria ma mai per ricercare pattern di co-movimento.
Per questo scopo è stato realizzato e successivamente adattato l'algoritmo CUTE.
CUTE consente di ricercare pattern di movimento su un numero custom di dimensioni mediante un approccio di mining ad alta dimensionalità.
CUTE inoltre è implementato sul framework spark (\cref{sec:bd:spark}).

In questo capitolo sarà presentato prima l'approccio del mining di itemset ad alta dimensionalità applicato ai dati di traiettoria (\textit{Colossal Trajectory Mining}).
In primo luogo verrà esposto il problema del \textit{Colossal Itemset Mining}, o ricerca di itemset su dati ad alta dimensionalità.
Successivamente si tratterà dell'algoritmo di Carpenter e delle applicazioni/limiti di questo approccio in letteratura.

Una volta fatto questo, si passerà alla formulazione dell'algoritmo CUTE e
del contributo apportato da questo nuovo approccio nell'ambito del clustering dei dati di traiettoria.

Di CUTE in primo luogo verranno presentate le esigenze che hanno portato alla stesura dell'algoritmo e delle novità dell'approccio rispetto allo stato dell'arte,
successivamente saranno presentate le fasi dell'algoritmo e i dettagli implementativi.
