L'idea alla base del \textit{Colossal Trajectory Mining},
o mining di traiettorie su larga scala, interseca entrambi gli ambiti del \textit{clustering} di traiettorie
e del \textit{Colossal Itemset Mining}

Il \textit{Colossal Itemset Mining} si pone come un'estensione del \textit{frequent itemset mining}
presentato nella~\nameCref{sec:frequent-itemset-mining}. Questa estensione riguarda l'utilizzo di dati
a elevata dimensionalità~\cite{zhu2007mining}, in alcuni domini è possibile infatti che il numero di attributi per ogni dato sia di molto superiore al numero stesso di dati.

Tale prospettiva può essere impiegata anche nell'analisi di dati di traiettoria. Considerando la superfice
spazio temporale coperta dall'insieme delle traiettorie e il numero di queste ultime, nella maggior parte dei casi risulterà evidente
che la dimensionalità di quest'ultimo dato sarà maggiore del precedente.
