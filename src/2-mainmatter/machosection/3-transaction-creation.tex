Nella prima fase viene prodotto un dataset nella forma \textit{<t\textsubscript{i}, (c\textsubscript{1},\ldots,c\textsubscript{n})>},
dove \textit{t\textsubscript{i}} rappresenta l'id di una certa traiettoria, mentre \textit{c\textsubscript{1},\ldots,c\textsubscript{n}}
l'insieme di celle in cui tale percorso è stato convertito.
Questo dataset tuttavia non è ancora adatto ad essere sottoposto a mining, poiché in questa situazione l'insieme delle transazioni è composto dalle traiettorie,
mentre quello delle feature dalle celle.
Un'operazione di ricerca di regole associative va ad individuare tra le feature quelle che compaiono assieme con almeno una certa frequenza,
quindi in questo caso il risultato dell'algoritmo sarebbe varii raggruppamenti di celle.
Questi insiemi possono essere sicuramente utili ad individuare percorsi frequenti e trafficati, ma non sono in grado di descrivere gruppi di movimenti comuni ad un certo insieme di oggetti.

Scopo di questo secondo passo di \textit{Cu.te} è quindi modificare il dataset ottenuto dalla prima fase
in modo da renderlo valido per la ricerca di pattern di movimento.
Per realizzare questa trasformazione, occorre invertire il ruolo di celle e traiettorie all'interno del dataset: prendendo come
ipotesi di invertire la relazione tra traiettoria e insieme di celle che la compongono, si ottiene la versione trasposta del dataset originale.
La versione invertita del database sarà quindi composta dalle singole celle del problema che ora saranno identificate
come le transazioni, mentre il ruolo delle traiettorie sarà di feature. Così facendo ad ogni cella sarà associato l'insieme delle traiettorie di cui
fa parte.
In questa situazione la ricerca di itemset frequenti produce raggruppamenti di traiettorie che hanno una frequenza maggiore
a una certa soglia di supporto.
Tale soglia può essere interpretata come il numero minimo di celle, ovvero di istanti spazio-temporali, che
l'intero gruppo di traiettorie deve aver condivisio per essere considerato interessante.
Alla luce di questa interpreatzione, diviene chiaro che questo vincolo svolge la stessa identica funzione del parametro \textit{k} definito nella~\cref{sec:problem:comovements-pattern},
per quanto riguarda invece \textit{m}, questo è espresso con  \(~\gamma \), presentato nella~\cref{subsec:cute:parameters}; infine \textit{g} è espresso come \(~\tau \) nella formalizzazione di \textit{Cu.Te}.

Invertendo il dataset è quindi possibile eseguire una ricerca di comovements pattern, ma l'inversione ha un'altra importante proprietà:
analizzando il rapporto tra il numero di celle generate nella prima fase e di traiettorie in tutto il dataset, è corretto affermare che
la quantità di queste ultime superi di gran lunga l'altra. La ricerca di pattern di movimento in città presenta infatti un'area spazio-temporale
ridotta rispetto agli oggetti che si muovono sopra questa superficie.
Di conseguenza il dataset che considera come transazioni le celle e come feature le traiettorie sarà etichettabile come dataset ad alta dimensionalità, adatto quindi a essere processato con le tecniche proprie del
\textit{Colossal Trajectory Mining}.


