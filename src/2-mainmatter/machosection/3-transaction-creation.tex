Nella prima fase viene prodotto un dataset nella forma \(\langle tr_{id}, (q_1,\ldots,q_n) \rangle\),
dove \(tr_{id}\) rappresenta l'id di una certa traiettoria, mentre \(q_1,\ldots,q_n\)
l'insieme di quanti in cui tale percorso è stato convertito.
Scopo di questo secondo passo di CUTE è quindi modificare il dataset ottenuto dalla prima fase in modo da renderlo valido per la ricerca di pattern di movimento.
Intuitivamente il dataset in output dalla prima fase non è ancora adatto ad essere sottoposto a mining, poiché in questa situazione l'insieme delle transazioni è composto dalle traiettorie,
mentre quello delle feature dai quanti.
Un'operazione di ricerca di itemset frequenti va ad individuare tra le feature quelle che compaiono assieme con almeno una certa frequenza,
quindi in questo caso il risultato dell'algoritmo sarebbe vari raggruppamenti di quanti.
Questi insiemi possono essere sicuramente utili ad individuare percorsi frequenti e trafficati, ma non sono in grado di descrivere gruppi di movimenti comuni ad un certo insieme di oggetti.
Come detto nella \cref{subsec:fim-trajectory}, per ottenere la ricerca di gruppi di oggetti bisognerebbe disporre di un dataset avente come transazioni i quanti e come feature le traiettorie.

Per ottenere un dataset adatto alla ricerca di gruppi, occorre trasporre il ruolo di quanti e traiettorie: prendendo come
ipotesi di trasporre la relazione tra traiettoria e insieme di quanti che la compongono, si ottiene la versione trasposta del dataset originale.
La versione trasposta del database sarà quindi composta dai singoli quanti del problema che ora saranno identificati come transazioni, mentre il ruolo delle traiettorie sarà di feature. 
Così facendo ad ogni quanto sarà associato l'insieme delle traiettorie in cui compare.
In questa situazione la ricerca di itemset frequenti produce raggruppamenti di traiettorie che hanno una frequenza maggiore a una certa soglia di supporto.
Questa frequenza può essere interpretata come il numero minimo di quanti condiviso tra tutti i componenti di un gruppo.

Trasponendo il dataset è quindi possibile eseguire una ricerca di gruppi di movimento, ma questa operazione ha anche un'altra importante proprietà.
Analizzando il rapporto tra il numero di quanti generati nella prima fase e di traiettorie in tutto il dataset, è corretto affermare che
la quantità di questi superi di gran lunga l'altra.
Di conseguenza il dataset che considera come transazioni le celle e come feature le traiettorie sarà etichettabile come dataset ad alta dimensionalità, adatto quindi a essere processato con le tecniche proprie del colossal itemset mining.

Queste operazioni sono eseguite alla lettera nella ricerca di pattern di movimento: ogni cella individuata diviene così una transazione del dataset utilizzato nella fase successiva.
