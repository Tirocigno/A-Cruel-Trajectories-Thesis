La prima fase di SPARE è la generazione degli Snapshot.
Durante questo passo viene valutata la dimensione spaziale delle traiettorie in relazione alla dimensione temporale.
Scopo della fase è individuare ad ogni istante quali oggetti sono vicini e quali no, in modo da creare una sequenza di cluster che descrivono l'evolvere dei gruppi nel tempo.

Unico passo di preprocessing necessario prima di questa fase è la definizione di una scala temporale univoca all'interno del dataset.
Tutti i punti devono infatti essere rapportati a una scala di tempo omogenea, altrimenti i passi successivi non sarebbero possibili.
Il primo passo consiste quindi nella definizione di una sequenza di istanti univoci chiamata \(T_{db}\).
Tutti i punti di ogni traiettoria saranno espressi rispetto a questa scala.

Una volta determinata la scala temporale, si scompone ogni traiettoria nei suoi singoli punti.
Questi punti vengono poi raggruppati sulla base dell'istante temporale in cui accadono.
Per ogni istante temporale nella scala sarà determinato un insieme di punti che indicano la posizione delle rispettive traiettorie al tempo \(t_{i} \in T_{db}\).

A questo punto è necessario valutare quali punti siano vicini e quali no.
Come detto nella \cref{sec:comovement-definition}, algoritmi di clustering diversi generano pattern di movimento diversi (density o distance based).
SPARE supporta la ricerca di entrambi i pattern: sta all'utente scegliere quale algoritmo impiegare.
Questa scelta risulta trasparente alle successive fasi di itemset mining.

Questo è l'unico passo dove vengono considerate le dimensioni spaziali.
Intuitivamente, ogni traiettoria avrà al massimo una posizione per ogni istante temporale.
Il processo di clustering risulterà quindi partizionante, così da prevenire eventuali ambiguità nelle fasi successive.

Questa operazione produce una serie di raggruppamenti di cluster.
Uno snapshot \(S_t\) non è altro che una coppia \(\langle t, S(t) \rangle\) dove \(t\) rappresenta un instante temporale e \(S(t)\) l'insieme dei cluster individuati a quello specifico istante.

Gli snapshot sono l'output di questa prima fase.