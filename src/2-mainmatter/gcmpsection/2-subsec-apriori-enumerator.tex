Ultimo passo dell'algoritmo: date in ingresso le varie stelle dirette prodotte al passo precedente, produce gruppi di movimento come itemset.
Intuitivamente questa fase riconosce tra le stelle della fase precedente quali coppie di nodi possono essere considerate come valide.
Una coppia di nodi sarà valida se la sequenza collegate risulterà significativa rispetto a \(k\), \(l\) e \(g\).
Tutte le coppie valide saranno poi fuse in itemset, intersecando ogni volta le sequenze temporali e assicurandosi che ad ogni passo l'intersezione rimanga valida.
Questo processo di esplorazione produce itemset massimali aventi dimensione maggiore di \(m\).

Passando poi al procedimento, si parte dalle stelle generate alla fine della fase precedente.
Ogni stella viene scomposta in coppie nella forma \(\langle(o_v, o_n):(t_1, \ldots, t_n)\rangle\).
\(o_v\) è il vertice della stella, \(o_n\) è uno dei nodi della stella, \((t_1, \ldots, t_n)\) sono gli istanti temporali sull'arco \((o_v, o_n)\).

Eseguendo una generazione e ricerca di itemset, si può pensare di applicare l'algoritmo Apriori (\cref{subsec:apriori}).
Tuttavia non è possibile applicare questo algoritmo.
Il principio della monotonicità tra un set e un suo superset non è valido nel caso degli itemset individuati da SPARE.

Si supponga di considerare due candidati \(C_1 = \langle(o_1, o_2): 1,2,3,6\rangle\) e \(C_2 = \langle(o_1, o_3): 1,2,3,7\rangle\).
Fissati \(m = 2, k = 3, l = 2, g = 1\), né \(C_1\) né \(C_2\) risultano itemset validi, violano infatti sia il vincolo su \(l\) che su \(g\).
In \(C_1,\; maxgap = (6-3) > g\) mentre per \(C2, \; maxgap = (7-3) > g\).
Per quanto riguarda \(l\) invece \(C_1 = \{C_1' = \langle(o_1, o_2): 1,2,3\rangle, C_1'' = \langle(o_1, o_2):6\rangle\}\), risulta evidente che \(|C_1''| < l\).
Altrettanto vale per \(C_2 = \{C_2' = \langle(o_1, o_3): 1,2,3\rangle, C_2'' = \langle(o_1, o_3):7\rangle\}\), anche qui \(|C_2''| < l\).
Definito \( C_3 = C_1 \cup C_2 = \langle(o_1, o_2, o_3): 1,2,3\rangle\), questo itemset risulta valido rispetto a tutti e quattro i parametri.
\(C_3\) quindi è un superset valido di due set non validi: ciò viola la definizione classica di monotonicità.

Occorre quindi definire una nuova nozione di monotonicità.
Per giungere a questo obbiettivo sono necessari tre nuovi concetti: \textit{Maximal G-Connected sequence}, \textit{Decomposable sequence} e infine \textit{Sequence semplification}.

Data una sequenza di istanti temporali \(T\), la massima sequenza connessa rispetto a \(g\) (MGS) è la più lunga sotto-sequenza \(T'\) tale che non \(T'\) risulti connessa rispetto a \(g\) e non esista \(T'' \supset T' \) che sia a sua volta connessa rispetto a \(g\).
Ogni MGS deve essere quindi connessa rispetto a \(g\) e massima nel numero di elementi.
Ogni sequenza può essere divisa in \(n\) MGS.
Ciascuna di queste MGS è disgiunta dalle altre se non per il primo o l'ultimo elemento.
Inoltre l'unione di tutte le MGS produce la sequenza originale.
Infine ogni MGS di una sotto-sequenza può essere considerata come sotto-sequenza di una MGS della sequenza originale.
Prese ad esempio le sequenze \(T' = (1,2,4,5,6,9,10,11,13)~\text{e}~T'' = (1,2,4,5,6)\) con \(g = 2\).
\(T'\) ha due MGS: \(T'_a = (1,2,4,5,6)\) e \(T'_b = (9,10,11,13)\).
\(T''\) ha una sola MGS corrispondente alla sequenza stessa.
\(T''\) è sotto-sequenza di \(T'\), quindi ogni MGS di \(T''\) dovrà essere sotto-sequenza di una MGS di \(T'\).
In questo caso, \(T''\) e \(T'_a\) coincidono, dimostrando la validità della proprietà.

Una volta definito il concetto di MGS, si può introdurre quello di sequenza decomponibile (\textit{Decomposable sequence}).
Una sequenza decomponibile è una sequenza che ha al suo interno almeno una sotto-sequenza valida rispetto a \(k\), \(l\) e \(g\).
Una sequenza si definisce decomponibile se, presa una qualunque delle sue MGS, questa risulta consecutiva rispetto a \(l\) e avente lunghezza maggiore o uguale a \(k\).
Prese ad esempio la sequenza \(T' = (1,2,4,5,6,9,10,11,13)\) con \(k=5\), \(l=2\) e \(g = 2\), si considerano le due MGS \(T'_a = (1,2,4,5,6)\) e \(T'_b = (9,10,11,13)\).
\(T'_a\) risulta sia consecutiva rispetto a \(l\) che lunga \(k\) elementi, di conseguenza \(T'\) può essere definito come sequenza decomponibile.

Infine si definisce semplificazione di sequenza (\textit{Sequence simplification}).
Data un sequenza \(T\) si definisce sequenza semplificata o \(Sim(T)\) la sotto-sequenza di \(T\) ottenuta applicando due trasformazioni:

\begin{itemize}
    \item \textbf{f-step}. Rimuove i segmenti di \(T\) che non rispettano la consecutività rispetto a \(l\).
    \item \textbf{g-step}. Tra le MGS individuate sul risultato di f-step, scarta quelle che hanno dimensione minore di \(k\). 
\end{itemize}

Ad esempio dato \(T = (1,2,4,5,6,9,10,11,13)\) e \(l = 2, k = 3, g = 2\).
Alla fine di f-step \(T = (1,2,4,5,6,9,10,11)\), \(13\) viene scartato in quanto facente parte di una sotto-sequenza di lunghezza uno. 
Questa nuova sequenza ha due MGS: \((1,2,4,5,6)\) e \((9,10,11)\).
Il secondo di questi due elementi viene scartato in quanto avente dimensioni minori di \(k\).
La sequenza semplificata risulta quindi \((1,2,4,5,6)\).

Il processo di semplificazione di una sequenza può portare alla creazione di una sequenza nulla, qualora non siano verificate certe condizioni su \(l\) e \(k\).
Per definizione, la sequenza vuota non risulta decomponibile.
Alla luce di ciò è possibile esprimere la seguente proprietà: 
per qualunque \(T\) supeset di una sequenza decomponibile vale che \(Sim(T) \neq \varnothing\).
Banalmente, per una sequenza decomponibile è impossibile che  \(Sim(T) \neq \varnothing\), ciò vale anche per ogni suo superset.

Una volta introdotto quest'ultimo concetto è possibile formulare la nuova definizione di monotonicità (\cref{definition:monotonicity}):

\begin{definition}[Monotonicità]\label{definition:monotonicity}

Dato un candidato pattern \(P = \langle O:T \rangle\), se \(Sim(P.T) = \varnothing\) allora ogni pattern \(P'\) tale che \(P' \supseteq P\) può essere scartato.

\end{definition}

La definizione si basa sulla considerazione che qualunque superset di \(P\) è composto dall'unione degli oggetti e dall'intersezione delle sequenze temporali.
Risulta a questo punto scontato affermare che qualunque sotto-sequenza di una sequenza tale che \(Sim(T) \neq \varnothing\) a sua volta produce una sequenza vuota durante il processo di semplificazione.

Definita la nuova monotonicità, lo pseudocodice della fasi di Apriori Enumerator è presentato nel \cref{alg:apriori-enumerator}.

\begin{algorithm}[H]
\caption{\( Apriori Enumerator\)}\label{alg:apriori-enumerator}
\begin{algorithmic}[1]
\Require \(Sr_s\)
\State \(C \gets \varnothing\)
\ForEach{ archi \(c = <o_i \cup o_j, T_i \cap T_j> \) in \(Sr_s\)} \Comment{Per ogni 2-itemset.}
        \If{\(Sim(T_i \cap T_j) \neq \varnothing\)} \Comment{Se la semplificazione dell'intersezione delle sequenze risulta valida}
            \State \(C \gets C \cup c\) \Comment{La tupla viene aggiunta a C} 
        \EndIf
    \EndFor
\State \(level \gets 2\)  

\While{\(C \neq \varnothing\)} 
    \ForEach{\(c_1 \in C\)}
        \ForEach{\(c_2 \in C \land |c_1.O \cup c_2.O| = level\)} \Comment{Per tutti i validi n-itemset}
            \State \(c' = <c_1.O \cup c_2.O, c_1.T \cap c_2.T>\) \Comment{Vengono generati tutti gli n+1 itemset}
                 \If{\( Sim(c_1.T \cap c_2.T) \neq \varnothing \)} \Comment{Se la semplificazione dell'intersezione delle sequenze risulta valida}
                    \State \(C \gets C \cup c\) \Comment{La tupla viene aggiunta a C} 
                \EndIf
        \EndFor
        \If{Nessun c' viene aggiunto a \(C\) e \(c_1\) è un pattern valido} 
            \State \Return \(c_1\) \Comment{\(c_1\) viene restituito in output in quanto massimale} 
        \EndIf
    \EndFor
    \State \(O_u \gets \) unione di tutti i \(c.O\) in \(C\)
    \State \(T_u \gets \) unione di tutti i \(c.T\) in \(C\)
    \If{\(Sim(T_u) \neq \varnothing\)} \Comment{Controllo della forward closure} 
            \State \Return \(<O_u, T_u>\) \Comment{L'intersezione tra tutti i possibili itemset in C è massimale, quindi l'esplorazione è terminata} 
        \EndIf
\EndWhile

\end{algorithmic}
\end{algorithm}

SPARE parte dalla valutazione degli itemset contenenti due elementi.
Questi vengono filtrati secondo il principio di monotonicità descritto sopra.

Successivamente comincia l'esplorazione dello spazio di ricerca: ogni itemset valido \(n\)-dimensionale viene fuso con tutti gli altri tali da formare 
itemset \(n+1\)-dimensionali.
Questi sono poi valutati con la stessa logica usata sopra: i validi vengono mantenuti, gli altri sono scartati.
Nel caso un itemset durante questa fase non produca nessun superset valido, questo viene valutato come output.
Se risulta rilevante rispetto a \(m\) e \(k\) viene aggiunto all'output, altrimenti viene scartato.

Una volta terminato di computare i candidati del livello successivo, viene eseguito un controllo sulla \(forward closure\).
Questa operazione consiste nel fondere assieme tutti gli itemset da valutare e testare se il candidato così costruito sia un itemset valido e rilevante.
Se ciò risulta vero, allora questo può essere restituito in output in quanto massimale, altrimenti i candidati singoli sono valutati con la stessa logica descritta sopra.

