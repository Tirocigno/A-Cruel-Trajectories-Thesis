Una volta definito che cos'è un dato di traiettoria, occorre mettere in chiaro che cosa distingue una traiettoria da un'altra.
Per confrontare due traiettorie, sarebbe sbagliato applicare le metriche di similarità per dati a bassa dimensionalità,
poiché gli oggetti in movimento producono dati complessi e con particolari correlazioni tra le dimensioni.
Il problema risulta quindi decisamente articolato: una buona metrica di similarità deve tenere conto non solo dei singoli punti, ma anche della
traiettoria nella sua interezza.
A questo va sommata la diversa lunghezza tra i due soggetti del confronto.
In letteratura sono presenti diversi metriche che possono essere impiegate per confrontare due traiettorie:

  La prima tra tutte le misure è la distanza euclidea. Grazie alla sua complessità lineare permette di gestire dati ad alta dimensionalità.
  Date due traiettorie  \(tr\textsubscript{i}\) e  \(tr\textsubscript{j}\) di lunghezza \(n\) e dimensioni \(p\), la loro distanza euclidea si definisce come:

  \begin{equation}
    {dist(tr_{i}, tr_{j}) = { {\frac{1}{n} } \sum_{k=1}^{n} {\sqrt{\sum_{m=1}^{p}{(a_{k}^{m} - b_{k}^{m})}^2}}}}
  \end{equation}

  La metrica tuttavia non è esente da difetti: è molto sensibile al rumore e richiede che le traiettorie siano uguali per numero di punti
  e dimensioni, inoltre anche l'intervallo di campionamento temporale deve coincidere.
  Questi limiti sono abbastanza difficili da aggirare quando si processa un dataset reale.

  Per superare questi problemi, sono disponibili diverse varianti della distanza euclidea: una su tutte è la \textit{Principal Component Analysis Plus Euclidean Distance} (PCA + distance)~\cite{zhang2006comparison}.
  Questa tecnica prima riduce le dimensioni spaziali ad una sola, successivamente esegue un'analisi PCA per convertire ogni traiettoria in
  \textit{k} coefficienti; a questo punto viene calcolata la distanza euclidea tra le traiettorie così trasformate.
  L'~\cref{def:PCA} mostra la formula per il calcolo della distanza PCA + distance: \(a_{k}^{c}\) e \(b_{k}^{c}\) rappresentano i
  \(k\) coefficienti nello spazio bidimensionale delle traiettorie \(tr_{i}\text{,}tr_{j}\). 
  \begin{equation} \label{def:PCA}
    {dist(tr_{i}, tr_{j}) = {\sqrt{\sum_{m=1}^{p}{(a_{k}^{c} - b_{k}^{c})}^2}}}
  \end{equation}
  Questa variazione mantiene gli stessi punti di forza della versione base della metrica, in più consente una maggior resistenza al rumore.

  La distanza di Hausdorff~\cite{chen2011clustering} è un alternativa a quella euclidea.
  Date due traiettorie, per ogni punto di una viene calcolato il più vicino punto dell'altra e la distanza tra i due.
  Il valore della distanza di Hausdorff
  corrisponde alla massima distanza calcolata nel passo precedente (~\cref{def:Hausdorff}).
  \begin{equation} \label{def:Hausdorff}
    \begin{multlined}
      {D(tr_{i}, tr_{j}) = \max{(h(tr_{i}, tr_{j}), h(tr_{j}, tr_{i}))}} \\
      ~where~{h(tr_{i}, tr_{j}) = \max_{a \in tr_{i}}{(\min_{b \in tr_{j}}{(dist(a,b))})}}
    \end{multlined}
  \end{equation}

  Nella formula \(h\) rappresenta la distanza di Hausdorff diretta, ovvero dai punti di una traiettoria verso l'altra; \(d\) invece sta per la distanza euclidea tra due punti.
  Il calcolo di entrambe le distanze dirette permette di gestire traiettorie con numero di punti differente tra loro.
  Questa metrica risulta robusta rispetto all'influenza causata da particolari distribuzioni di punti, ma allo stesso tempo è sensibile al rumore.
  
  Longest Common Sub Sequence~\cite{rick2000efficient} affronta il problema con un approccio diverso: invece che calcolare la distanza fra i punti delle due traiettorie, computa la
  più lunga sotto-sequenza comune ad entrambe.
  La lunghezza di quest'ultima determina la vicinanza: più il valore è alto, quindi la sotto-sequenza aumenta di dimensioni, più le due traiettorie sono vicine.
  Essendo impossibile una coincidenza assoluta dei punti tra due traiettorie sono definite due soglie, \(\epsilon \) e \( \sigma \) che rispettivamente
  modellano la tolleranza rispetto all'asse x e y.
  L' \cref{def:lcss} definisce la metrica in termini formali: 

  \begin{equation} \label{def:lcss}
        D(tr_{i}, tr_{j}) =
        \begin{cases}
         0 & n = m = 0 \\
         1 + LCSS_{\epsilon\sigma}(Head(tr_{i}), Head(tr_{j})) & 
         |a_{i}^{x} - a_{j}^{x}| \leq \epsilon 
         \land
         |a_{i}^{x} - a_{j}^{x}| \leq \sigma \\
         \begin{aligned}
             \max{(
             LCSS_{\epsilon~\sigma}{(tr_i, Head(tr_j))}, \\
             LCSS_{\epsilon~\sigma}{(Head(tr_{i}), tr_{j})}
             )}
         \end{aligned}
         & altrimenti 
      \end{cases}
  \end{equation}

  
  LCSS può essere calcolata in modo ricorsivo, inoltre consente una certa tolleranza rispetto
  alle deviazioni nei dati, ciò consente una buona efficienza nei dataset reali.
  Il maggior limite della metrica sta nella definizione dei parametri
  \(\epsilon \) e \( \sigma \) in problemi complessi.

  Dynamic time warping (DWT)~\cite{chen2005robust} pone il focus sulla dimensione temporale rispetto a quelle spaziali, come accadeva nelle metriche precedenti.
  Lo scopo è trovare l'allineamento ottimo tra due traiettorie dati certi vincoli.
  DWT resiste bene alle differenti lunghezze tra traiettorie: obbiettivo della misura è infatti ricercare il percorso a cui assimilare le traiettorie che abbia il minor coefficiente di distorsione
  calcolato sulle trasformazioni subite dalle traiettorie.
  
    \begin{equation} \label{def:dwt}
     { D_{d}(tr_{i}, tr_{j}) =
        \begin{cases}
         0 & n = m = 0 \\
         \infty & n = 0~||~m = 0 \\
         dist(a_{i}^{k} - a_{j}^{k}) +
         \min{
         \begin{cases}
         D_{d}(Rest(tr_{i}), Rest(tr_{j}))) \\
         D_{d}(tr_{i}, Rest(tr_{j})) \\
         D_{d}(Rest(tr_{i}),tr_{j}) 
         \end{cases}
         } & altrimenti
      \end{cases}
      }
  \end{equation}
  
  L'~\cref{def:dwt} definisce DWT: la distanza è calcolata in maniera ricorsiva sommando ad ogni passo la distanza degli elementi corrispondenti delle due sequenze.
  La distanza \(dist\) rappresenta la distanza euclidea tra due punti mentre \(Rest\) indica il segmento di traiettoria non ancora esplorato.
  Il termine di questo calcolo avviene quando entrambe le traiettorie sono state esplorate.
  DWT assicura il rispetto dell'ordine tra i punti delle traiettorie, inoltre introducendo un principio di scalabilità locale della dimensione temporale,
  riesce a gestire scale temporali differenti tra le traiettorie.
  DWT tuttavia richiede continuità all'interno dei punti, è sensibile al rumore e a sotto-traiettorie molto distanti tra loro.

  Infine si tratta della
   metrica di Fréchet~\cite{khoshaein2013trajectory}, la quale considera in contemporanea sia la dimensione temporale
  che quella spaziale.
  Date due traiettorie di uguale lunghezza \(n\), si calcola la distanza euclidea tra i punti aventi stessa posizione all'interno delle due traiettorie:
  il valore più alto corrisponde alla distanza di Fréchet.
  La formula per calcolare la distanza di Fréchet 
  è mostrata nell'~\cref{def:frechet}:
  
    \begin{equation} \label{def:frechet}
    \begin{multlined}
      {D_{f}(tr_{i}, tr_{j}) = 
      \min{||C||}}~\text{dove}~||C|| =
      \max{}_{k=1}^{K}{(dist(a_{i}^{k},
      b_{j}^{k}))}
    \end{multlined}
  \end{equation}
  
  \(D_{f}\) rappresenta la misura della distanza, \(K\)
  è il valore minore di lunghezza tra \(tr_{i}\) e \(tr_{j}\), \(a_{i}^{k}~\text{e}~
      b_{j}^{k}\) sono i \(k\)-esimi punti di \(tr_{i} \text{,} tr_{j}\) e infine \(dist\) è la misura della distanza euclidea.
  Qualora le due traiettorie divergano come dimensioni, si eseguirà questo calcolo su tutte le possibili sotto-traiettorie di lunghezza \(n\) generabili
  dalla traiettoria più lunga.
  La distanza di Fréchet considera la traiettoria nella sua continuità, per questo motivo è molto sensibile agli outlier.
  
Le diverse misure della distanza hanno costi computazionali differenti tra di loro, che sono riassunti nella~\cref{tab:metric-cost}. 

\begin{table}[H]
    \centering
   \begin{tabular}{||c c||}
 \hline
 Misura & Costo computazionale \\ [0.5ex] 
 \hline\hline
 Euclidea & O(\(n\)) \\ 
 \hline
  PCA + Euclidea & O(\(n\)) \\ 
 \hline
  Hausdorff & O(\(n*m\)) \\ 
 \hline
 LCSS & O(\(n*m\)) \\ 
 \hline
 DWT & O(\(n*m\)) \\ 
 \hline
 Fréchet & O(\(n*m\)) \\ 
 \hline
\end{tabular}
    \caption{Costi computazionali delle metriche di similarità}
    \label{tab:metric-cost}
\end{table}

\begin{center}

\end{center}
