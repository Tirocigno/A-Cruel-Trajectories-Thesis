
Come detto nella definizione di traiettoria (\cref{subsec:trajectory-definition})
le informazioni necessarie per definire un punto sono due: la componente spaziale e quella temporale.

Il tempo risulta più complesso da gestire rispetto allo spazio: è infatti praticamente impossibile
definire una scala temporale univoca all'interno di un dataset.
Essendo le traiettorie generate da diversi dispositivi GPS, è molto raro che questi condividano tra loro la frequenza
di campionamento, rendendo quindi difficile definire un ordine assoluto all'interno dell'area temporale
coperta dal dataset.
Oltre a ciò, la possibile adozione di scale cicliche per l'analisi del tempo rende necessario
introdurre ulteriore complessità negli algoritmi che supportano queste ricerche.

A differenza di quanto accade negli ambiti spaziali, la ricerca pone il suo accento
sull'interpretazione del tempo e la conseguente formazione dei cluster piuttosto che
solo su questo secondo ambito.

La maggior parte degli algoritmi riescono a gestire scale temporali assolute. Ad esempio
\textit{T-OPTICS}~\cite{nanni2006time} è una variante di OPTICS che impiega una metrica di similarità
adatta a individuare cluster considerando anche il tempo.
L'idea alla base dell'algoritmo è di ricercare il miglior intervallo temporale l'algoritmo OPTICS, appositamente modificato per la ricerca di traiettorie, individua i risultati migliori.
Sta all'utente specificare la lunghezza e il range dell'intervallo temporale: a seconda del periodo specificato i cluster individuati possono cambiare totalmente.
Come però affermato in un'indagine~\cite{mitsch2013survey} condotta nel 2013, esistono pochi framework in grado di gestire
scale temporali cicliche e la ricerca di pattern periodici.
