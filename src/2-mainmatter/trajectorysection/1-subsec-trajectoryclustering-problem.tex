
Come detto nella~\cref{sec:measure}, i dati di traiettoria sono molto più complessi
rispetto ai dati solitamente utilizzati con gli algoritmi di clustering tradizionali.
Occorre quindi definire modifiche di questi ultimi per riuscire a fare operazioni di clustering
efficienti, in quanto sarebbe impossibile catturare tutta la complessità dell'informazione con
tecniche pensate per dati a bassa dimensionalità.

Alla luce di quanto detto sopra, per il clustering di traiettorie sono individuabili i seguenti obbiettivi:

\begin{itemize}

  \item \textbf{Supporto alla dimensionalità dei dati}.
  Obbiettivo del clustering di traiettorie è la ricerca di cluster tenendo conto di tutte le informazioni presenti sui dati.
  Ognuno di questi attributi dovrà essere considerato nel momento in cui verranno portate avanti le operazioni di divisione e raggruppamento delle traiettorie.

  \item \textbf{Definizione di una metrica di similarità tra traiettorie}.
  Come presentato nella~\cref{sec:measure} il problema della similarità tra traiettorie è complesso e sono presenti diverse soluzioni.
  Scopo della ricerca è quello di individuare metriche che individuino le differenze tra le traiettorie in maniera affidabile e efficace.

  \item \textbf{Qualità dell'algoritmo}.
  L'algoritmo utilizzato nelle operazioni di clustering deve essere efficiente e scalabile, ad esempio impiegando apposite strutture dati per
  ridurre i tempi di accesso ai dati, oppure utilizzando le tecnologie di computazione Big Data per velocizzare l'esecuzione degli algoritmi.

\end{itemize}

Nonostante nessuno degli algoritmi di clustering tradizionali abbia tutte le caratteristiche espresse sopra, le idee alla loro base rimangono comunque
valide in buona parte dei casi.
Di conseguenza molti algoritmi pensati per i dati di traiettoria non sono che estensioni di quelli già noti in letteratura.

Gli algoritmi di clustering di traiettorie sono 
classificabili secondo la natura del dato in output
e sulla tipologia di clustering.
La natura del dato è direttamente collegata alle dimensioni considerate nelle operazioni di clustering:
la ricerca può essere condotta considerando solo la componente spaziale o includendo anche quella temporale.
La tipologia di clustering invece riguarda i cluster prodotti in output: algoritmi partizionanti 
produrranno cluster disgiunti, algoritmi di clustering sovrapposto cluster la cui intersezione non è vuota.

La \cref{tab:clus-alg} riassume i principali algoritmi che saranno trattati nelle sezioni successive alla luce della classificazione appena introdotta.
\begin{table}[H]
    \centering
   \begin{tabular}{||c c c||}
 \hline
 Algoritmo & Natura dei dati & Tipologia di clustering \\ [0.5ex] 
 \hline\hline
CB-SMoT & Spaziale & Sovrapposto \\ 
 \hline
CACT & Spaziale & Sovrapposto \\ 
 \hline
T-OPTICS & Spazio-temporale & Sovrapposto \\
 \hline
TraceMob & Spaziale & Partizionante \\
 \hline
 DSC & Spazio-temporale & Sovrapposto \\
 \hline

\end{tabular}
    \caption{Classificazione degli algoritmi di clustering di traiettorie trattati}
    \label{tab:clus-alg}
\end{table}

\begin{center}

\end{center}
