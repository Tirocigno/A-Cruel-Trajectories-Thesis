Il clustering di traiettorie è uno degli argomenti più discussi all'interno del mining di traiettorie, che mira
a individuare le similarità tra i movimenti dei vari oggetti e dedurre comportamenti comuni sulla
base di una divisione in cluster.

Come detto nella~\cref*{subsec:problem:trajectorydata}, i dati di traiettoria sono molto più complessi
rispetto ai dati solitamente utilizzati con gli algoritmi presentati nella~\cref*{subsec:problem:clustering}.
Occorre quindi definire modifiche degli algoritmi presenti in letteratura per riuscire a fare operazioni di clustering
efficienti, in quanto sarebbe impossibile catturare tutta la complessità dell'informazione con
tecniche pensate per dati a bassa dimensionalità.

Alla luce di quanto detto sopra, per il clustering di traiettorie sono individuabili i seguenti obbiettivi:

\begin{itemize}

  \item Supporto alla dimensionalità dei dati.
  Obbiettivo del clustering di traiettorie è la definizione di cluster tenendo conto di tutte le informazioni presenti sui dati.
  Ognuno di questi attributi dovrà essere considerato nel momento in cui verranno portate avanti le operazioni di divisione e raggruppamento delle traiettorie.

  \item Definizione di una metrica di similarità tra traiettorie.
  Come presentato nella~\cref{subsec:problem:trajectorydata} il problema della similarità tra traiettorie è complesso e sono presenti diverse soluzioni.
  Scopo della ricerca è quello di individuare metriche che individuino le differenze tra le traiettorie in maniera affidabile e efficace.

  \item Qualità dell'algoritmo.
  L'algoritmo utilizzato nelle operazioni di clustering deve essere efficiente e scalabile, ad esempio impiegando apposite strutture dati per
  ridurre i tempi di accesso ai dati, oppure utilizzando le tecnologie di computazione Big Data per velocizzare l'esecuzione degli algoritmi.

\end{itemize}

Nonostante nessuno degli algoritmi di clustering tradizionali abbia tutte le caratteristiche espresse sopra, le idee alla loro base rimangono comunque
valide in buona parte dei casi.
Di conseguenza molti algoritmi pensati per i dati di traiettoria non sono che estensioni di quelli già noti in letteratura.

Analizzando il panorama dei possibili algoritmi di clustering di traiettorie, si possono individuare alcune famiglie sulla base delle caratteristiche a cui
viene dato maggior peso: gli algoritmi spaziali, quelli temporali, il clustering basato sul partizionamento, il clustering di traiettorie incerte e infine algoritmi
basati su un raggruppamento di natura semantica.
