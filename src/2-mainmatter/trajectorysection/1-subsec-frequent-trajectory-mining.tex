La ricerca di itemset frequenti può essere adattata ai dati di traiettoria.
Ciò non è altro che un'evoluzione del mining di sequenze.

Il mining di pattern su un insieme di item è un caso particolare di mining di itemset frequenti.
La principale modifica è l'introduzione del concetto di tempo e , conseguentemente, di sequenza.
Una sequenza non è altro che un itemset i cui elementi sono ordinati rispetto al tempo.
Ad esempio, l'itemset \(<a, ab, c>\) può essere interpretato come una sequenza in cui i tre elementi
si susseguono in istanti temporali contigui.
Nella ricerca di sequenze, gli attributi sono espliciti.
Nel caso dei dati di traiettoria però, ciò non risulta vero: ogni punto di traiettoria grezza è composto da latitudine, longitudine e istante temporale.
Tali punti sono difficilmente coincidenti tra loro, ciò causa una difficoltà nell'estrazione delle feature.
Per superare questa problematica, occorre ricorrere all'utilizzo di astrazioni di traiettoria:
mappando il dataset su uno specifico sistema di riferimento, è possibile accomunare i punti negli stessi quanti.

Questa astrazione consente di superare certi limiti, tuttavia non è sufficiente a risolvere il problema.
Gli algoritmi di riconoscimento di itemset frequenti o sequenze, oltre ad avere un alto costo computazionale, non sono in grado di processare la contiguità spaziale e la continuità temporale.
Inoltre nel riconoscimento di sequenze sono richieste strette relazioni di adiacenza tra le feature, cosa non sempre possibile nelle traiettoria, a causa della loro intrinseca incertezza.

In letteratura sono state proposte diverse soluzioni per superare questi limiti:
ad esempio per quanto riguarda la generazione delle sequenze è stato proposto l'utilizzo di una struttura ad albero per generare i pattern di movimento \cite{cao2005mining}.
Per quanto riguarda invece le relazioni ignorate, sono stati proposti approcci per considerare la contiguità spaziale \cite{chen2011personal} e quella temporale \cite{lv2015route}. 

Le applicazioni del mining di traiettorie presenti in letteratura riguardano l'estrazione dei percorsi 
più frequentati \cite{qiu2016mining} e delle regioni più trafficate \cite{zheng2018spatial}.
È interessante considerare come la ricerca di luoghi e percorsi frequentati sia ortogonale rispetto all'individuazione di gruppi di oggetti che hanno viaggiato assieme per un certo periodo di tempo.
Approfondendo questo confronto, emerge che la definizione del problema è molto simile, ma cambia il ruolo di transazioni e feature.
Nel caso di ricerca di luoghi frequenti in un dataset, tale frequenza è determinata sulla base degli oggetti che passano per un certo punto.
Trattando invece di analisi di oggetti che si sono mossi assieme, 
intuitivamente il supporto è definibile come l'insieme di posizioni condivise dal gruppo perché sia considerato frequente.

La ricerca di gruppi di oggetti non è affrontabile con le ordinarie tecniche di itemset mining.
Il numero degli oggetti infatti supera di gran lunga quello delle posizioni visitate, di conseguenza 
l'insieme delle transazioni avrebbe cardinalità molto inferiore rispetto a quello delle feature.
Tale condizione esplode il costo computazionale per la ricerca di itemset frequenti.