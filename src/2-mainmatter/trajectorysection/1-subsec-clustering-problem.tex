Il \textit{Clustering}, o analisi di raggruppamento, è una tecnica non supervisionata con lo scopo di aggregare i dati in \textit{cluster} o gruppi
tali che i dati all'interno di un gruppo siano più simili tra loro rispetto a quelli all'esterno\cite{liao2005clustering, zazzarro2009clustering}.

I metodi di \textit{clustering} per elaborare dati statici di differenti tipologie sono cinque:

\begin{itemize}
  \item Metodi basati sul partizionamento:
  \item Metodi basati sulla gerarchia:
  \item Metodi basati sulla densità:
  \item Metdi basati su una griglia:
  \item metodi basati sul modello:
\end{itemize}


