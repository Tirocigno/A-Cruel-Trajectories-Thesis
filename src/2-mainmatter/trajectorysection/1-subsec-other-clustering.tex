Nelle sezioni precedenti sono stati analizzati i dati di traiettoria sotto il profilo
spaziale e temporale, tuttavia sarebbe inesatto limitare la panoramica sul clustering di traiettorie
a queste due categorie.
Lo spazio-tempo sono indubbiamente le dimensioni più importanti all'interno di una traiettoria,
tuttavia come affermato nella~\cref{subsec:problem:trajectorydata},
i dati che costituiscono quest'ultima hanno molte più dimensioni di quelle fin d'ora considerate.

Partendo da questo punto, sono state ideate altre categorie di clustering e raggruppamento dei dati
che sfruttano parte di queste informazioni in combinazione con gli attributi spazio-temporali,
che mantengono comunque un ruolo cruciale nella divisione delle traiettorie.

Una traiettoria è spesso composta da un grande numero di punti e processata nella sua completezza.
Questo però comporta che molti algoritmi tendano a ignorare caratteristiche locali delle traiettorie
e di conseguenza mancare il riconoscimento di similarità tra diverse sotto-traiettorie.
Per risolvere questo problema, sono stati ideati algoritmi per scomporre le traiettorie in segmenti più corti
e eseguire operazioni di clustering utilizzando queste sotto-traiettorie.
Questa scomposizione non rende solo più agevoli le operazioni di clustering, ma cattura anche
quelle particolarità che sarebbero scartate processando le traiettorie nella loro interezza.
Gli algoritmi basati su questa idea sono definiti \textit{Partition and group based algorithm}.
Il focus principale di questa categoria è l'individuazione dei segmenti e
dei punti in cui ``spezzare'' la traiettoria originale.
Per risolvere questo problema sono state ipotizzate diverse soluzioni,
ad esempio il framework \textit{DST} (Distribuited Subtrajectory Clustering)~\cite{tampakis2019scalable} utilizza una metrica basata
sul cambio di densità nell'intorno dei punti della traiettoria per determinare le divisioni.
In generale questi framework hanno buoni risultati quando computano dataset di traiettorie
di lunghezze varie, inoltre resistono molto bene all'aumento del numero di punti da processare;
tuttavia non è banale individuare i segmenti ideali per catturare tutte le peculiarità
di una traiettoria.

Una traiettoria per definizione è composta da un insieme limitato di punti, tale numero
è direttamente collegato alla frequenza di campionamento del dispositivo GPS che registra
il moto in questione.
Nel caso in cui questa frequenza sia particolarmente alta, può succedere che si venga a creare
un certo grado di incertezza all'interno della traiettoria stessa: dati due punti consecutivi
\(p_{1}\) e \(p_{2}\) registrati agli istanti \(t_{1}\) e \(t_{2}\), non c'è modo di sapere quali
movimenti abbia eseguito l'oggetto tra  \(t_{1}\) e \(t_{2}\).
Questa mancanza di informazioni ha dato origine a una categoria di algoritmi,
chiamati \textit{Uncertain Trajectory Clustering algorithm}, con lo scopo direttamente
creare cluster tenendo conto di questa variabilità nel singolo dato.
L'introduzione dell'incertezza all'interno di questi framework consente loro di poter
processare anche dataset in cui sono presenti numerosi outlier e i dati in generale
hanno molto rumore.
Un altra applicazione di questa categoria è nella deanonimizzazione dei dati: grazie
alla ricerca di cluster incerti è possibile invertire in parte il processo di anonimizzazione
dei dati all'interno di un dataset e dedurre così informazioni nascoste dal dataset, come
ad esempio il gruppo di appartenenza di un certo oggetto sulla base dei suoi movimenti.

Nessuno degli approcci fino ad ora trattati ha posto la propria attenzione sulle informazioni non
strettamente spazio-temporali, come ad esempio la misura della velocità o accellerazione di un oggetto
o la direzione del suo movimento.
Gli algoritmi basati sulla semantica mettono in primo piano queste informazioni rispetto a quelle
spazio-temporali, ottenendo risultati altrimenti impossibili da ottenere.
Ad esempio considerando velocità e accellerazione è possibile determinare quali possano
essere i punti in cui una certa categoria di utenti tende a fermarsi più spesso~\cite{zheng2008understanding},
oppure capire quale sarà il percorso di un utente assegnato a una certa categoria sulla base
di come si sono mossi gli altri apparteneti al medesimo gruppo~\cite{ying2011semantic}.
Questa categoria è in continua crescita, con algoritmi sempre più complessi e che sfruttano sempre di
più l'interzza dell'informazione offerta dal singolo dato.







