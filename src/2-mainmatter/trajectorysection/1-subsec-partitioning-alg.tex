Gli algoritmi di clustering, oltre alla dimensioni impiegate nella ricerca, possono 
essere classificati sulla base delle proprietà dei cluster prodotti.
Tra tutte le proprietà una delle più interessanti è sicuramente la natura partizionante o sovrapposta:
si definisce un algoritmo partizionante se, dato un punto, questo viene assegnato al massimo a un cluster,
sovrapposto se questa cardinalità aumenta.
Anche gli algoritmi di clustering di traiettorie possono essere classificati in partizionanti e sovrapposti.

Il clustering di traiettorie partizionante produce insiemi disgiunti, effettuando quindi una separazione 
totale tra i percorsi nel dataset.
Ogni traiettoria è valutata nel suo complesso dal processo di clustering, ciò implica che vengano avvicinati
i percorsi che sono più complessivamente simili mentre siano distanziati quelli che condividono solo brevi tratti in comune.

\textit{TraceMob} (Transformation, partition clustering and cluster evaluation of moving objects) è un esempio di algoritmo partizionante.
L'idea alla base del framework è di dividere lo spazio coperto dalle traiettorie in celle di dimensioni 
\(\alpha * \beta\). 
Questi parametri determinano la scala dell'analisi: valori alti producono aree grandi, adatte a territori in cui le traiettorie sono molto sparse, valori piccoli invece sono idonei per la ricerca in spazi ad alta densità.
Successivamente viene calcolata la vicinanza tra tutti gli elementi del dataset: due traiettorie risulteranno vicine se in tutto il loro percorso sono sempre transitate in celle vicine.
Una volta terminato il calcolo, ogni traiettoria viene proiettata come punto su uno spazio \(d\)-dimensionale.
La funzione che si occupa di ciò è strutturata in modo da avvicinare le traiettorie simili e separare quelle diverse.
Infine viene effettuato un clustering partizionante sui punti così generati. 
Tale operazione produrrà cluster di traiettorie vicine nella loro interezza, separando quelle che 
divergono in certi istanti.

Il clustering partizionante considera quindi le traiettorie nella loro totalità.
Ciò può risultare vantaggioso in termini di performance, tuttavia questa modalità 
ignora le singole caratteristiche locali della traiettoria.
