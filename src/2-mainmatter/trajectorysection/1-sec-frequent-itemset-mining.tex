Nell'ambito dell'analisi dei dati, il \textit{frequent itemset mining}, o ricerca di itemset frequenti,
è uno dei task con maggiori ambiti di applicazione: può essere impiegato nei processi di classificazione,
clustering o ricerca di outlier~\cite{article}.

Il frequent itemset mining necessita di un insieme di transazioni per effettuare la propria ricerca,
ogni transazione, o riga (\textit{row}) che dir si voglia, è composta poi da un insieme di attributi,
o \text{feature}, che identificano gli elementi all'interno della singola transazione.
Occorre specificare che di una feature è considerata rilevante ai fini della ricerca solo la presenza
o l'assenza e non un eventuale valore o altre prorietà collegate; queste potranno essere integrate
mediante apposite tecniche di preprocessing.

Scopo del mining di itemset frequenti è di ricercare, dato un insieme di transazioni, ricercare
le combinazioni di feature, o itemset, che risultano frequenti.
Misura fondamentale per definire la frequenza è il supporto: tale metrica può essere
descritta come il numero di transazioni che contengono un certo itemset.
All'interno della ricerca di itemset frequenti viene definito un limite inferiore al supporto
per determinare l'interesse verso un certo itemset; questo parametro prende il nome
di supporto minimo, o \textit{minsup}.
Definito ciò, è possibile esprimere il problema del mining di itemset frequenti con
la seguente formulazione (~\cref{definition:fim}) :

\begin{definition}[Frequent itemset mining]\label{definition:fim}
  Definito \(T\) come l'insieme delle transazioni e \(F\) come quello delle feature complessive,
  il frequent itemset mining ricerca tutti gli itemset \( i = \{ f_{1}, \ldots, f_{n}\}, f_{1}, \ldots, f_{n} \in F\)
  tali che definito il supporto \( S = \{ t_{1}, \ldots, t_{m} \} \in T \; s.t \; \forall t_{i} \in \{ t_{1}, \ldots, t_{m} \}
  i \subseteq t_{i}  \), \(|S| \geq minsup\)
\end{definition}

Definito l'obbiettivo della ricerca, sono necessari due passaggi per realizzarla:
il primo passo consiste nella generazione di tutti i possibili itemset, il secondo
nel calcolo del supporto per ciascuno di questi e \textit{pruning} (potatura) dei candidati
non interessanti.
Nonostante la definizione semplice, la complessità computazionale di queste fasi può esplodere:
supponendo infatti di avere un set di feature contenente \(n\) diversi elementi, l'insieme di tutte
le possibili combinazioni generabili è \(2^n\), rendendo di fatto molto costoso il processo
di ricerca in presenza di dataset di larghe dimensioni.

