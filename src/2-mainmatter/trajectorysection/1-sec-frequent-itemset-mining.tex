Nell'ambito dell'analisi dei dati, il \textit{frequent itemset mining}, o ricerca di itemset frequenti,
è uno dei task con maggiori ambiti di applicazione: può essere impiegato nei processi di classificazione,
clustering o ricerca di outlier~\cite{article}.

Il frequent itemset mining necessita di un insieme di transazioni per effettuare la propria ricerca,
ogni transazione, o riga (\textit{row}), è composta poi da un insieme di attributi,
o \text{feature}, che identificano gli elementi all'interno della singola transazione.
Occorre specificare che di una feature è considerata rilevante ai fini della ricerca solo la presenza
o l'assenza e non un eventuale valore o altre proprietà collegate; queste potranno essere integrate
mediante apposite tecniche di preprocessing.
La definizione di transazione e item è formalizzata nella \cref{definition:transaction-attribute}

\begin{definition}[Transazione, item e itemset]\label{definition:transaction-attribute}
  Dato un database, si definisce transazione \(t\) la singola riga della base di dati.
  Tale riga è caratterizzata da un identificatore e da un insieme di item o \(feature\).
  Un item \(i\), è un attributo binario, calcolato sulle colonne del database da cui sono estratte le transazioni.
  Si definisce infine un itemset \(I\) un set di item \(i_1, \ldots, i_n\).
\end{definition}


Scopo del mining di itemset frequenti è di ricercare, dato un insieme di transazioni, ricercare
le combinazioni di feature, o itemset, che risultano frequenti.
Misura fondamentale per definire la frequenza è il supporto: tale metrica può essere
descritta come il numero di transazioni che contengono un certo itemset (\cref{definition:support}).

\begin{definition}[Supporto]\label{definition:support}
Definito un insieme di transazioni \(T\) e un certo itemset \(I\) si definisce supporto \(s(I)\) l'insieme \((t_1, \ldots, t_n) \in T\) 
tale che \(I \subseteq t_i.I~\forall t_i \in (t_1, \ldots, t_n)\).
Il valore del supporto è la cardinalità dell'insieme \((t_1, \ldots, t_n)\)
\end{definition}

All'interno della ricerca di itemset frequenti viene definito un limite inferiore al supporto
per determinare l'interesse verso un certo itemset; questo parametro prende il nome
di supporto minimo, o \textit{minsup}.
Definito ciò, è possibile esprimere il problema del mining di itemset frequenti con
la seguente formulazione (~\cref{definition:fim}) :

\begin{definition}[Frequent Itemset Mining]\label{definition:fim}
  Definito \(T\) come l'insieme delle transazioni e \(F\) come quello delle feature complessive,
  il frequent itemset mining ricerca tutti gli itemset \\
  \( I = \{ i_{1}, \ldots, i_{n}\}, i_{1}, \ldots, i_{n} \in F\)
  tali che definito il supporto \( S = \{ t_{1}, \ldots, t_{m} \} \in T \; s.t \; \forall t_{i} \in \{ t_{1}, \ldots, t_{m} \}
  \; i \subseteq t_{i}  \), \(|S| \geq minsup\)
\end{definition}

Definito l'obbiettivo della ricerca, sono necessari due passaggi per realizzarla:
il primo passo consiste nella generazione di tutti i possibili itemset, il secondo
nel calcolo del supporto per ciascuno di questi e \textit{pruning} (potatura) dei candidati
non interessanti.
Nonostante la definizione semplice, la complessità computazionale di queste fasi può esplodere:
supponendo infatti di avere un set di feature contenente \(n\) diversi elementi, l'insieme di tutte
le possibili combinazioni generabili è \(2^n\), rendendo di fatto molto costoso il processo
di ricerca in presenza di dataset di larghe dimensioni.

