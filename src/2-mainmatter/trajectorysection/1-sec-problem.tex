Uno degli obiettivi dell'analisi di dati di traiettoria è il \textit{clustering} (raggruppamento) di traiettorie simili.
Una traiettoria può essere considerata non solo come il tragitto percorso dall'oggetto che la genera,
ma anche come l'insieme delle attività, ciascuna corrispondente a una posizione, eseguite dal medesimo oggetto.
Lo scopo del \textit{clustering} di traiettorie è quindi di scoprire quali condividono una certa similarità e quali invece no.
In tal modo è possibile identificare eventuali raggruppamenti di oggetti che hanno percorso assieme una certa frazione del loro percorso, oppure individuare eventuali
traiettorie differenti da tutte le altre.
Interpretando poi questi risultati partendo dalle traiettorie come insiemi di attività, è possibile ad esempio dedurre quali possano essere attività comuni e quali invece no.
