Uno degli obiettivi dell'analisi di dati di traiettoria è il \textit{clustering} (raggruppamento) di traiettorie simili.
Una traiettoria può essere considerata non solo come il tragitto percorso dall'oggetto che la genera,
ma anche come l'insieme delle attività, ciascuna corrispondente a una posizione, quando a ogni posizione si collega un significato semantico.

Lo scopo del clustering di traiettorie è quindi di scoprire quali percorsi condividono una certa similarità e quali invece no.
In tal modo è possibile identificare eventuali raggruppamenti di oggetti che hanno viaggiato assieme per una certa frazione del loro percorso, oppure individuare
traiettorie differenti da tutte le altre.

Interpretando invece una traiettoria come l'insieme dei comportamenti di un oggetto,
è possibile ad esempio dedurre quali possano essere attività comuni e quali invece no.
