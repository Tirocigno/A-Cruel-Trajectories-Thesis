Nelle sezioni precedenti sono state prese in considerazione le principali categorie di clustering di dati di traiettoria; queste sono state impiegate in
diversi ambiti, a seconda delle singole caratteristiche della tecnica utilizzata.

Di seguito sono riportati i principali ambiti applicativi del clustering di traiettorie:

\begin{itemize}

  \item \textbf{Analisi dei pattern di movimento degli oggetti.}
  L'applicazione più scontata di un raggruppamento di traiettorie: l'analisi di pattern di
  movimento mira ad individuare quali oggetti si sono mossi assieme secondo certe regole.

  \item \textbf{Previsione dei movimenti di un oggetto.}
  Sulla base delle caratteristiche di una traiettoria, è possibile predire quali saranno
  i futuri movimenti di un oggetto partendo dai suoi spostamenti passati: si attribuisce infatti
  il percorso fin d'ora eseguito a uno dei cluster individuati e considerando le sue caratteristiche
  si può prevedere con una certa accuratezza il percorso futuro dell'oggetto.
  La validità di questa previsione è direttamente collegata al numero di misure considerate durante
  le operazioni di clustering: maggiore il numero, più accurato il risultato.
  Importanza particolare ha il tempo: non considerandolo è di fatto impossibile eseguire questo tipo
  di ricerca.

  \item \textbf{Supporto alla pianificazione stradale e dei trasporti.}
  Come affermato nella~\cref{subsec:problem:spatialalgorithms}, determinati algoritmi
  possono individuare i punti più frequentati da certi utenti; è quindi possibile integrare queste
  informazioni in piani di controllo del traffico/ espansione delle infrastrutture urbane.

  \item \textbf{Ricerca di outlier.}
  Un outlier è per definizione un elemento che, dato un certo criterio di similarità,
  si discosta da tutti gli altri. Gli outlier possono essere interpretati come errori, ma anche
  come comportamenti che per determinati motivi divergono dalla norma.
  Sotto quest'ultima chiave di lettura divengono interessanti la loro ricerca e i motivi per cui
  differiscono dagli altri dati.
  Essendo poi il percorso di un oggetto interpretabile come l'insieme dei suoi comportamenti,
  gli outlier individuano comportamenti inusuali e per questo possono destare grande interesse.

  \item \textbf{Deanonimizzazione dei dati.}
  Utilizzando tecniche di clustering su dati incerti, è possibile ricavare informazioni sugli
  oggetti collegati alle traiettorie precedentemente nascosti tramite tecniche di anonimizzazione.

\end{itemize}

Gli ambiti di utilizzo del clustering di traiettorie sono numerosi, tuttavia il margine di miglioramento
è ancora abbondante.
In primo luogo la maggior parte degli algoritmi non sfrutta tutto il potenziale semantico dei dati:
molte tecniche impiegano solo le dimensioni spazio-temporali o un insieme ristretto di feature,
scartando così informazioni che potrebbero ulteriormente migliorare la qualità dei risultati.
Successivamente i risultati prodotti sono di difficile interpretazione, molto spesso accade che
l'estrazione di conoscenza mostri relazioni banali o al contrario troppo complesse per essere spiegate.
Infine manca ancora un'integrazione diffusa con le tecnologie di elaborazione Big Data,
la maggior parte degli algoritmi infatti è pensata per computare in maniera centralizzata, rinunciando
ai vantaggi del calcolo distribuito.
In questo modo larghi dataset producono risultati di scarsa
qualità in tempi alti.
L'applicazione di queste strategie di computazione, sebbene complessa nella sua realizzazione, porterebbe
notevoli vantaggi e supererebbe molte delle problematiche degli attuali algoritmi.
I framework \textit{G.C.M.P}\cite{DBLP:journals/pvldb/FanZWT16} e \textit{C.U.T.E} pongono questo
supporto come uno dei loro punti di forza.


