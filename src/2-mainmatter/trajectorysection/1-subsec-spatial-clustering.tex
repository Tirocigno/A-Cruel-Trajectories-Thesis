Le informazioni spaziali sono probabilmente la feature più importanti all'interno di un dato di traiettoria.
Analizzando come un oggetto si muove e i luoghi che visita possono essere ricavate una vasta serie di informazioni.
Negli anni vari algoritmi sono stati proposti per estrarre dai dati informazioni differenti tra loro.

Il primo ambito di ricerca sui dati spaziali riguarda i luoghi di maggior interesse, ovvero le posizioni
in cui sono passati un certo numero di oggetti all'interno del dataset.
A questa categoria appartiene il framework \textit{CB-SMoT} (Clustering Based Stop and Moves of Trajectories)~\cite{palma2008clustering}.
Questo algoritmo ricerca all'interno delle varie traiettorie considerando gli stop, ovvero segmenti in cui la velocità della traiettoria cala sotto una certa soglia o risulta uguale a zero.
Successivamente questi segmenti sono raggruppati in cluster usando una versione modificata di DBSCAN\@.
Tale versione dell'algoritmo si basa sulla velocità invece che sulla densità.
Infine ogni cluster viene confrontato con la mappa dell'area coperta dalle traiettorie e viene associato a uno specifico punto o area.
Questa associazione permette di interpretare meglio il risultato ottenuto dall'algoritmo.


Ancora è possibile estrarre da un insieme di dati di traiettorie, l'insieme delle strade più
frequentate; ciò diverge dal primo ambito presentato poiché la ricerca di un percorso risulta
più complessa rispetto a quella di un singolo punto: una strada infatti ha caratteristiche molto
più complesse di una singola località, come ad esempio una continuità nello spazio tra i vari punti che
la compongono.
\textit{CACT}~\cite{hung2015clustering} (Clustering and Aggregating Clues of Trajectories) è un possibile framework per
ricercare percorsi che rappresentino i comportamenti di una certa categoria di utenti.
L'idea dell'algoritmo è di definire una misura di similarità basata sugli indizi (\textit{clue}):
Un indizio è definibile come la vicinanza spazio-temporale di punti di traiettorie diverse che però condividono lo stesso comportamento.
Tale indizio costituisce una corrispondenza parziale di comportamento.
Sulla base della presenza di indizi simili, vengono costituiti cluster di traiettorie, che raggruppano queste ultime sulla base di un certo comportamento.
I cluster così ottenuti sono però ancora percorsi parziali, per determinare percorsi completi è necessario un ulteriore passo di ricerca di indizi e fusione dei cluster.

I due ambiti appena descritti partono dalla stessa interpretazione dei dati, cercando di eseguire una separazione tra le varie traiettorie sulla base delle proprietà dei singoli punti.
Un'alternativa a questa visione è presente nel clustering basato su forma (\textit{Shape Based Clustering}),
in cui i raggruppamenti sono basati sulla distribuzione dei punti piuttosto che sulle loro proprietà.
Questo approccio non si limita ad analizzare solo la dimensione spaziale, ma include nel determinare la forma di una traiettoria anche la sua dimensione temporale.
