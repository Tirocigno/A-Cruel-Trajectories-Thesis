Il contributo fornito da questa tesi si spinge nell'analisi approfondita del clustering di traiettorie e del frequent itemset mining.
Durante questa analisi è stata individuata una categoria a metà tra il clustering e il frequent trajectory mining: la ricerca di pattern di co-movimento.
Sono stati indagati quindi i principali algoritmi di analisi di questi pattern.
Tra tutti il framework SPARE è risultato particolarmente interessante, per via della sua implementazione distribuita.
Inoltre SPARE formalizza il concetto e la ricerca di pattern di co-movimento generici (GCMP), pattern che tramite la configurazione di alcuni parametri possono esprimere i principali pattern di co-movimento.
SPARE tuttavia risulta limitato nel momento in cui si vuole espandere la dimensionalità dei dati oppure si considerano vincoli di contiguità nello spazio.

In questa tesi è stato implementato il framework CUTE, algoritmo generico per la ricerca di gruppi di traiettorie su un insieme di dimensioni custom in ambito big data.
Questo è stato adattato e configurato per la ricerca di co-movement pattern.
CUTE è stato poi testato approfonditamente su vari dataset per valutare i suoi risultati e le sue performance.
Traendo qualche conclusione dal lavoro svolto, è possibile dire che la ricerca di pattern di movimento è una categoria ampia e gli algoritmi presenti in letteratura faticano a coprila nel suo intero.
Esistono infatti una serie di possibilità e definizioni di pattern che difficilmente sono totalmente compatibili tra di loro.
CUTE lavora in questa direzione, tentando di coprire tutti i pattern di movimento presenti nella categoria.
La chiave per questo sta nell'approccio di CUTE al problema della ricerca: in primo luogo la possibilità di esprimere un sistema di riferimento custom consente di modellare la ricerca di un pattern su un qualunque insieme di dimensioni, monotone o meno.
Successivamente la natura generica dei pattern restituiti in output consente di ricercare più categorie sulla base della configurazione di parametri specificata.
In conclusione CUTE è un framework decisamente valido per cercare di affrontare il problema della ricerca di pattern di co-movimento nella sua totalità.

CUTE però non si limita alla ricerca di pattern di co-movimento, ma può essere espanso alla ricerca di altre categorie di gruppi.
Durante gli esperimento compiuti, non sono mai considerate dimensioni diverse dallo spazio e dal tempo.
In futuro sarebbe sicuramente utile provare ad integrare altre dimensioni nella ricerca e vedere i risultati estratti dall'algoritmo.
Un'altra possibile direzione riguarda l'aggiunta di ulteriori meccaniche di pruning per gli itemset individuati.
Com'è possibile notare dai risultati, il numero di cluster per parametri ragionevoli è comunque molto alto.
Qualora sia necessaria una ricerca più fine, si potrebbe pensare ad esempio di introdurre una nuova metrica, la coesione, per valutare la bontà di un itemset.
La coesione esprimerebbe il rapporto tra i movimenti di un gruppo e i movimenti dei singoli elementi che lo compongono.
Cluster abbastanza coesi rappresenterebbero oggetti che hanno viaggiato assieme per buona parte del loro percorso singolo.
Al contrario cluster con un basso valore di coesione sarebbero composti da oggetti che hanno viaggiato assieme una breve parte del loro percorso totale, di conseguenza sarebbero meno interessanti.