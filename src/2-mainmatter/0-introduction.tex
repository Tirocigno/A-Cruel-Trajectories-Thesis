La sempre più grande abbondanza di dispositivi in grado di monitorare la posizione degli oggetti ha portato a un'esplosione nelle quantità e qualità di dati di traiettorie.
Enormi quantità di traiettorie, nella forma di sequenze spazio-temporali di punti, sono registrate ogni momento da chip di controllo di animali, dispositivi GPS su veicoli o su
piattaforme wearable. 
Questa grande abbondanza di dati rende possibili diverse tipologie di analisi su dati di traiettoria, come ad esempio la pianificazione del traffico, l'analisi di movimenti di animali e profilazione di un utente sulla base dei suoi movimenti.
Questi sono solo alcuni esempi di applicazioni del trajectory data mining.
Scopo di questo ambito del data mining è l'applicazione di algoritmi per estrarre informazione da dati di traiettoria.
La conoscenza così creata può poi essere impiegata per il miglioramento della qualità della vita del singolo individuo o a supporto delle decisioni in ambito governativo o industriale.

Fra gli scenari più interessanti dell'analisi di traiettorie è l'identificazione di pattern di co-movimento.
Si definisce pattern di movimento un gruppo di oggetti che hanno viaggiato assieme per un certo periodo di tempo.
Questa ricerca è direttamente collegata al clustering di traiettorie, il cui obbiettivo è suddividere un insieme di traiettorie in gruppi secondo un determinato criterio.
Nel caso in cui il criterio di similarità si concentri sui luoghi visitati, allora sarà possibile dedurre dai raggruppamenti quali possono essere le strade o i luoghi più frequentati.
Se invece l'analisi si pone sugli oggetti e i gruppi che formano nel tempo, allora sarà possibile identificare certe categorie di utenti che viaggiano assieme e i loro comportamenti comuni.

Il lavoro presentato in questa tesi può essere diviso in quattro passi.
In primo luogo è stato condotto uno studio sullo stato dell'arte del trajectory data mining, soffermandosi sulla definizione di clustering e frequent itemset mining, prestando particolare attenzione alle loro applicazioni all'interno dell'analisi di traiettorie.

Successivamente è stata trattata la ricerca di pattern di co-movimento, con particolare attenzione al framework SPARE.
Di questo framework, sviluppato in ottica big data, sono state analizzati a fondo funzionamento, potenzialità e limiti.

Una volta terminata questa analisi è stato condotto lo studio e la realizzazione di un nuovo framework, chiamato CUTE, per la ricerca di pattern di movimento.
A differenza di quanto presente in letteratura, CUTE propone un nuovo approccio distribuito basato sulla divisione dello spazio di ricerca in celle sulla base di un sistema di riferimento arbitrario.
Lo spazio di ricerca in questione viene composto da un numero custom di dimensioni.
A questo proposito non ci sono vincoli: possono essere impiegate solo dimensioni spazio-temporali, solo semantiche o un misto tra le due categorie.
Su queste viene poi eseguita una ricerca mediante tecniche di mining di dati ad alta dimensionalità.

Infine sono stati confrontati i due framework sopracitati, analizzando pregi e svantaggi in determinate situazioni tramite un confronto sui risultati ottenuti.

Alla luce di questo, il documento è strutturato come segue:
nel \cref{chapter:chapter1} vengono fornite la definizione di traiettoria e dei concetti ad essa collegati, successivamente è analizzato il problema del clustering.
Questo viene poi declinato nelle applicazione relative ai dati di traiettoria.
Ciò avviene analogamente anche per il frequent itemset mining.
Nel \cref{chapter:chapter2} sono brevemente elencate le tecnologie utilizzate durante lo sviluppo dei due framework oggetto della tesi.
Il \cref{chapter:chapter3} presenta il problema della ricerca dei pattern di movimento, successivamente espone il framework SPARE per l'individuazione di questi pattern.
Il problema del mining di traiettorie basato su dati ad alta dimensionalità e la sua applicazione nell'algoritmo CUTE sono oggetto del \cref{chapter:chapter4}.
Il \cref{chapter:chapter5} mostra quali sono stati i test effettuati e i dataset utilizzati per questi, sono inoltre presenti le interpretazioni dei risultati tra i confronti di SPARE e CUTE.
Nel \cref{chapter:chapter6} infine sono tratte alcune conclusioni sul lavoro svolto e potenziali punti di partenza per ricerche successive.