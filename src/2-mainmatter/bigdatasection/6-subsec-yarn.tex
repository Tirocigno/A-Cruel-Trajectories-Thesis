YARN (Yet Another Resource Negotiator) è il gestore di risorse introdotto con Hadoop 2.0.
Originalmente sviluppato per migliorare le performance di MapReduce, YARN risulta abbastanza generico per adattarsi anche ad altri paradigmi.
Compito di questo gestore di risorse è negoziare le componenti necessarie per eseguire una certa applicazione e gestire la computazione distribuita.
YARN rimane trasparente all'utente: viene usato solamente da framework e mai direttamente dal codice dell'utente.

L'idea alla base di YARN è la separazione tra la gestione dei job e quella delle risorse.
Questa divisione è ulteriormente accentuata dalla presenza di due demoni che si occupano di questi aspetti.
YARN si compone di due processi demone:

\begin{itemize}
    \item \textbf{Resource Manager (RM)}.
    Demone globale, ne esiste solo uno per cluster, gestisce le risorse tra le varie applicazioni.
    Si compone di \textit{Scheduler} e \textit{Application Manager}.
    Il primo si occupa di allocare le risorse per le varie applicazioni, mentre il secondo è responsabile di accettare i job e eventualmente farli ripartire in caso di errori.
    \item \textbf{Node Manager (NM)}.
    Uno per ogni nodo slave, si occupa di eseguire i container.
    Un container è un'entità usata per eseguire un processo con un insieme limitato di risorse.
    Il Node Manager monitora l'esecuzione dei processi, il loro utilizzo di risorse e riporta i dati sul Resource Manager.
\end{itemize}

Nel momento in cui viene inviata al sistema la richiesta dell'esecuzione di una nuova applicazione, il RM cerca tra i vari NM uno che possa lanciare l'\textit{Application Master Process}.
A questo punto l'Application Manager negozia il primo container per eseguire l'AMP, che poi provvederà a richiedere le risorse necessarie.
Questo processo risulta altamente scalabile: RM infatti delega ai singoli AMP la richiesta delle risorse e di nuovi container.
In caso di fallimento inoltre è l'Application Master a occuparsi del ripristino.

YARN risulta quindi un framework aperto, che supporta altri paradigmi oltre a Map-Reduce, inoltre si integra bene con i meccanismi nativi di HDFS.